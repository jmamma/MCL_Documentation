\chapter{MCL Sequencer Pages:}

\textit{The primary Sequencer Pages are accessed using the \textbf{<Page>} button.} These include:
\begin{itemize}
    \item (MD) Step Editor Page
    \item (Ext MIDI) PianoRoll Editor Page
    \item (MD/Ext MIDI) Chromatic Page
\end{itemize}
\section{Track selection}
The current track in any sequencer page can be chosen by pressing \textbf{<Shift>} to enter the Track Menu and modifying the TRACK SELECT option.
\\\\
For the Step Editor page the current track selection is automatically synced to the Machinedrum. When you change track on the MD the track will change on the MegaCommand.\\\\
Alternatively, if a MIDI device is connected to port 2 and you are in either the Chromatic or PianoRoll pages, the track will automatically change to the first Ext MIDI track that is set to the same channel as incoming note data.\\\\
From the Chromatic or PianoRoll page, by pressing MD \textbf{[Trig]} keys 1-6 whilst the Track menu is open will change the focused Ext MIDI track.
\section{Live Record:}
\textit{Live Record mode can be activated from any Sequencer page by pressing \textbf{<Save >}.\\Alternatively, hold \textbf{[REC]} and press \textbf{[Play]} on the MD to activate record mode.}\\\\Depending upon the page type, Live Record can be used to capture:
\begin{itemize}
    \item Trig presses
    \item Parameter changes, CC Automation
    \item Notes played in Chromatic Mode
    \item Notes played, Pitch Bend + Channel Pressure on the PianoRoll Editor page.
\end{itemize}
\textit{The MD's \textbf{[Clear]} function can be used to clear the recorded sequence for the current track.}

\newpage
\section{Track menu}
\screenshot{track_menu.png}

\textit{From within a sequencer page, the track menu can be opened by holding \textbf{<Shift>}.\\The selected entry is activated upon release.}\\\\
The track menu consists of the following entries that are common to all Sequencer Pages:

\begin{figure}[hb]
    \begin{tabular}{|l|l|}
    \hline
    \rowcolor[HTML]{C0C0C0} 
    Entry            & Function                                                        \\ \hline
    Edit     & change editor mode   \\ \hline
    Track Select     & select active track \\ \hline
    Copy             & \begin{tabular}[c]{@{}l@{}}TRK: copy track, ALL: copy pattern\end{tabular}                                             \\ \hline
    Clear            & \begin{tabular}[c]{@{}l@{}}TRK: clear track, ALL: clear pattern\end{tabular}                                           \\ \hline
    Paste            & \begin{tabular}[c]{@{}l@{}}TRK: paste track, ALL: paste pattern\end{tabular}                                           \\ \hline
    Speed & \begin{tabular}[c]{@{}l@{}}The current track's playback speed\\ 1x, 2x, 3/2x, 3/4x, 1/2x, 1/4x, 1/8x.\\ Hold <Load> and release <Shift> to change speed of all tracks. \end{tabular}                          \\ \hline
    Shift            & \begin{tabular}[c]{@{}l@{}}L/R: shifts the track left/right.\\ L ALL/R ALL: shifts the pattern left/right.\end{tabular} \\ \hline
    Reverse          & TRK: reverse the track, ALL: reverse the pattern  \\ \hline
    Rec Quant        & ON: quantize record enabled, OFF: quantize record disabled \\ \hline
    \end{tabular}
\end{figure}