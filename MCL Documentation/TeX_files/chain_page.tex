\chapter{Load Page}
The Load page is used to load slots from the Grid. When a slot is loaded, Machine/Sound data is sent to the corresponding MIDI device and the sequencer data is loaded into the matching sequencer track.\\
\\
\screenshot{load_from_a.png}\\
\textit{The Load Page is accessible from the GridPage by pressing the  \textbf{<Load>} function button or by pressing the MD's \textbf{[YES]} key.}
\encodersbuttons{Mode: [ MAN, AUT, QUE ]}{Length Override: [ -, N ]}{--}{ Quantization}{Cancel Load}{--}{--}{Group Select}

\newpage
\section{Loading Individual Tracks}
The Load Page utilises the MD's \textbf{[Trig]} keys to specify which slots are to be loaded. Pressing multiple \textbf{[Trig]} keys and then releasing them will load the corresponding slots from the current row of the visible Grid.
\section{Grid Toggle}
When in the Save or Load page, the \textbf{<Shift 1>} button can be used to toggle between Grid X or Grid Y.
\section{Simultaneous Load from Grid X and Grid Y}
It is possible to simultaneously load a collection of tracks from both Grids X and Y. 
\begin{itemize}
\item First select the tracks from Grid X pressing and holding the corresponding \textbf{[Trig]} keys.
\item Tap \textbf{<Shift 1>} to switch grids
\item Release the \textbf{[Trig]} selection
\item Select tracks from the alternate Grid Y pressing and holding the corresponding \textbf{[Trig]} keys. 
\item Finally release the second grid \textbf{[Trig]} selection to confirm the action. 
\end{itemize}

\section{Load Track Groups:}
When in the Save or Load Page, holding the \textbf{<Shift 2>} button or MD's \textbf{[YES]} key opens the Group Select menu,
allowing you to load or save all tracks corresponding to a group.\\An entire row/pattern across both Grids X + Y can be saved this way.\\
\\
There are four groups:
\begin{enumerate}
    \item MIDI Device 1 (MD)
    \item MIDI Device 2 (A4/MNM/Generic MIDI)
    \item FX (MDFX + LFOTrack + RouteTrack)
    \item TEMPO
\end{enumerate}
From the Group Select Menu each group can be enabled/disabled using the MD's \textbf{[Trig]} keys.\\
\\
Releasing \textbf{<Shift 2> / [YES]} will save/load tracks corresponding to the active groups.ve/load tracks corresponding to the active groups.
\newpage
\section{Load mode:}
Each column of the grid has a dedicated MODE setting. The MODE setting is applied to the matching column on load, and can be set to one of the following values:

\begin{itemize}
    \item Manual (MAN):  The selected slots will be loaded at the next transition interval. Existing queues in matching columns will be discarded.
    \item Auto (AUT): The selected slots will be loaded at the next transition interval. If the slots LOOP setting is greater than 0, the column will begin loading slots automatically based on their LOOP/JUMP settings. Existing queues in matching columns will be discarded.
    \item Queue (QUE): Each slot will be added to the corresponding column's playback queue. The slots in the queue will be loaded sequentially and repeat in a loop. A maximum of 8 slots can be queued per column. Queued slots are drawn with inverse colours on the grid.
\end{itemize}

As each column has an independent mode setting, it is possible to have some slots loading automatically (Auto) according to their Loop/Jump setting, other slots can be looping in an improvised queue (Queue) the remaining slots could be left static via manual chain (Manual).

\section{Slot Length Override:}
The LEN paramater allows overriding the loaded slot's length. This can be useful if you have a short track, of say 4 steps, and instead want it to play for 16, looping throughout. Alternatively if you are loading a group of slots, each with different length, it is possible to force them to all play to the same length.

\section{Quantization Rules}
The quantization rules specify the transition interval for the selected slots.\\
\\
The minimum Quantization value is 4 steps. Quantization values can be changed by adjusting the "Q:" parameter and increase in powers of 2.\\
\begin{itemize}
\item 04: Toggle cue on next possible 4th step
\item 08: Toggle cue on next possible 8th step 
\item 16: Toggle cue on next possible 16th step 
\item 32: Toggle cue on next possible 32th step 
\item 64: Toggle cue on next possible 64th step
\end{itemize}

\section{Gain staging}
To provide consistent level mixing across loads, the LEV parameter of a loaded track is never transmitted when the sequencer is running. 

\newpage
\chapter{Additional Loading:}
\section{Loading from Grid}
\section{Loading using Bank + Pattern Select}
