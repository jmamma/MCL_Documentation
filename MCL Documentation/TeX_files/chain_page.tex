\chapter{Load Page}
The Load page is used to load slots from the Grid. When a slot is loaded, Machine/Sound data is sent to the corresponding MIDI device and the sequencer data is loaded into the matching sequencer track.\\
\\
\screenshot{load_from_a.png}\\
\textit{The Load Page is accessible from the GridPage by pressing the  \textbf{<Load>} function button or by pressing the MD's \textbf{[YES]} key.}
\encodersbuttons{Load MODE: [ MAN, AUT, QUE ]}{Length Override: [ -, N ]}{--}{ Quantization}{Cancel Load}{--}{--}{Group Select}\\
\textit{The Length and Quantization encoders can be toggled between exponential and incremental value changes by holding down the encoder button when rotating.}
\\\\
The MD's \textbf{[Bank A/B/C]} keys can be used to quickly change the load MODE setting between [ MAN, AUT, QUE ].
\newpage
\section{Loading Individual Tracks}
The Load Page utilises the MD's \textbf{[Trig]} keys to specify which slots are to be loaded. Pressing and then releasing multiple \textbf{[Trig]} keys will load the corresponding slots from the current row of the visible Grid.
\section{Grid Toggle}
When in the Save or Load page, the \textbf{<Shift>} button can be used to toggle between Grid X or Grid Y.
\section{Simultaneous Load from Grid X and Grid Y}
It is possible to simultaneously load a collection of tracks from both Grids X and Y. 
\begin{itemize}
\item First select the tracks from Grid X pressing and holding the corresponding \textbf{[Trig]} keys.
\item Tap \textbf{<Shift>} to switch grids
\item Release the \textbf{[Trig]} selection
\item Select tracks from the alternate Grid Y pressing and holding the corresponding \textbf{[Trig]} keys. 
\item Finally release the second grid \textbf{[Trig]} selection to confirm the action. 
\end{itemize}

\section{Load Track Groups:}
When in the Save or Load Page, holding the \textbf{<Shift>} button or MD's \textbf{[YES]} key opens the Group Select menu,
allowing you to load or save all tracks corresponding to a group.\\An entire row/pattern across both Grids X + Y can be loaded or saved this way.\\
\\
There are four groups:
\begin{enumerate}
    \item MIDI Device 1 (MD)
    \item MIDI Device 2 (A4/MNM/Generic MIDI)
    \item FX (MDFX + LFOTrack + RouteTrack)
    \item TEMPO
\end{enumerate}
From the Group Select Menu each group can be enabled/disabled using the MD's \textbf{[Trig]} keys.\\
\\
Releasing \textbf{<Shift> / [YES]} will load tracks corresponding to the active groups.
\newpage

\section{Playback Queue:}
Each column of the grid has a dedicated Playback Queue.\\\\Up to 8 slots can be queued in each column.  The slots in the queue form a repeating chain, each slot will be loaded sequentially. When the end of the queue is reached the first slot is loaded again.\\\\Queued slots are drawn with inverse colours on the Grid Page.\\\\The Playback Queue for any column can be activated by setting the load MODE to QUE and then loading a slot in that column.

\section{LOOP \& JUMP:}
Each slot contains a LOOP \& JUMP setting which can be adjusted via the Grid page's Slot Menu.\\\\The LOOPS value specifies how many times a track should repeat before it transitions to the slot located at the the bank \& row specified by JUMP.
\\\\The LOOP \& JUMP settings only applies when a slot is loaded with load MODE set to: Auto (AUT). Auto mode is a useful way to create preset arrangement or songs.

\section{Load MODE setting:}
Each column of the grid has a dedicated MODE setting. The MODE setting is applied to the corresponding column on load, and can be set to one of the following values:

\begin{itemize}
    \item Manual (MAN):  The selected slots will be loaded at the next transition interval. Existing playback queues in matching columns will be discarded.
    \item Auto (AUT): The selected slots will be loaded at the next transition interval. If the slots LOOP setting is greater than 0, the column will begin loading slots automatically based on their LOOP/JUMP settings. Existing playback queues in matching columns will be discarded.
    \item Queue (QUE): Each slot will be added to the corresponding column's playback queue. The slots in the queue will be loaded sequentially and repeat in a loop. A maximum of 8 slots can be queued per column. 
\end{itemize}

As each column has an independent mode setting, it is possible to have some slots loading automatically (AUT) according to their LOOP \& JUMP setting, other slots can be looping in an improvised queue (QUE) the remaining slots could be left static via manual load (MAN).
\newpage
\section{Slot Length Override:}
When load mode is set to QUE, the LEN parameter allows overriding the loaded slot's length. This can be useful if you have a short track, of say 4 steps, and instead want it to play for 16, looping throughout. Alternatively if you are loading a group of slots, each with different length, it is possible to force them to all play to the same length.

\section{Quantization Rules:}
The quantization rules specify the transition interval for the selected slots.\\
The minimum quantization value is 2 steps. Quantization values can be changed by adjusting the "Q:" parameter and increase in powers of 2. Holding down the encoder button will step increment the quantization value.
\begin{itemize}
\item 02: Toggle cue on next possible 2nd step
\item 04: Toggle cue on next possible 4th step
\item 08: Toggle cue on next possible 8th step 
\item 16: Toggle cue on next possible 16th step 
\item 32: Toggle cue on next possible 32th step 
\item 64: Toggle cue on next possible 64th step
\end{itemize}

\section{Track Levels:}
\textit{To provide consistent level mixing across slot loading, the LEV parameter of a loaded track is never transmitted when the sequencer is running.}\\\\This allows the performer to use the LEV parameter to fade the loaded track in and out of the mix. The MD's track's VOL parameter should therefor be used to control the relative track loudness.\\\\
\textit{Please read the manual section describing the \textbf{Config-->Machinedrum-->Normalize} setting which automatically adjusts a saved track's gain staging for this purpose.}

\chapter{Other Loading Methods:}
\section{Loading via [Bank] + [Trig]:}
The Machinedrum's pattern changing functionality has been replicated in MCL. Each row of the Grid can be imagined as a pattern, compromising of slots for each of the MD's 16 tracks, plus Auxiliary slots containing the Master FX settings.\\
\\
\textit{A combination of \textbf{[Bank]} + \textbf{[Trig]} will load the corresponding row of the grid.\\\\
When multiple \textbf{[Trig]} keys are pressed, rows are chained together. This is achieved by automatically setting each column's load MODE setting to QUE and then adding slots to each Playback Queue.}
\\\\
When loading via \textbf{[Bank] + [Trig]} slots are loaded from the chosen row, according to the Load Page's Group Selection settings. If both the MD and FX groups are active, this is effectively the same behaviour as pattern loading on the MD.

\section{Loading via the Grid:}
From the Grid Page, pressing and holding the \textbf{<Shift>/[NO]} button will open the Slot Menu. The MD's \textbf{[Up/Down/Left/Right]} keys can then be used to highlight a selection of slots.\\\\
\textit{When the Slot Menu is active, selected slots can be loaded by pressing the MD's \textbf{[YES]} key.}
\\\\Slots are loaded according to the current load MODE setting. If more than one row is selected, the load MODE is temporarily set to QUE, and the loaded slots are added to each column's Playback Queue.
\\\\
\textit{The MD's \textbf{[Bank A/B/C]} keys can be used to quickly change the load MODE setting between [ MAN, AUT, QUE ].}