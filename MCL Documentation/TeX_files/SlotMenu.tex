\chapter{Slot Menu:}

The Slot Menu is used to manipulate the properties of selected slots; to clear, copy, paste one or more slots and to configure a slot's link settings. It also provides an option to rename the current row and to alternate the visible Grid between X or Y.
\screenshot{slot_menu.png}
\\
\textit{The Slot Menu is accessible from the GridPage by holding down the \textbf{<Shift>} function button or the MD's \textbf{[No]} key.\\\\When \textbf{<Shift>} / \textbf{[No]} is released, any changes to the menu will be applied to the selected slots. }
\section{Multiple Slot Selection:}
A rectangular selection containing multiple slots can be made by entering the slot menu and rotating \textbf{< Encoder 3>} and \textbf{<Encoder 4>}. Alternatively the MD's \textbf{[Up/Down/Left/Right]} arrow keys can be used to make a selection.
\screenshot{range_copy.png}
\section{Slot loading}
When the Slot Menu is active, selected slots can be loaded by pressing the MD's \textbf{[YES]} key. 
Slots are loaded according to the current load MODE setting. If more than one row is selected, the losf MODE is automatically set to QUE, and the loaded slots are added to each column's playback queue.
\\\\
\textit{Slot loading is covered in more detail in the "Saving and Loading" section of this manual.}
\newpage
\section{Slot Menu Options:}
\begin{tabular}{|l|l|}
\hline
\rowcolor[HTML]{C0C0C0} 
Entry                                  & Function                                                                       \\ \hline
Grid: {[} X, Y {]}                     & switch between Grid X or Y.                                                      \\ \hline
Mode: {[} –, AUT, MAN, QUE {]} & \begin{tabular}[c]{@{}l@{}}AUT: sets load mode to auto. \\ MAN: sets load mode to manual.\\ QUE: sets load mode to queue.\end{tabular} \\ \hline
Len: (0, 64)                           & step length of track/slot.\\
                                           \\ \hline
Loops: (0, 64)                         & how many times to loop track/slot.\\
                                           \\ \hline
Jump: (0,127)                           & \begin{tabular}[c]{@{}l@{}}which row the current slot is to\\ load/jump to after n loops.\end{tabular}                                                                               \\ \hline

Clear: {[} –, YES {]}                  & clear the selected slot(s).                                                                                                                                                                   \\ \hline
Copy: {[} –, YES {]}                   & copy the selected slot(s).                                                                                                                                                                    \\ \hline
Paste: {[} –, YES {]}                  & paste the selected slot(s).                                                                                                                                                                   \\ \hline
Rename                                 & rename the current row on Grids A and B.                                                                                                                                                       \\ \hline
\end{tabular}
