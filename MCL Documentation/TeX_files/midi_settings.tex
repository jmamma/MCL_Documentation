\chapter{MIDI Settings Menu:}
The MIDI menu provides access to seven MIDI configuration sub-menus.
\screenshot{midi_menu.png}
\\
\textit{Changes made in each MIDI configuration are applied upon menu exit.}

\section{Port Config:}
The Port Config menu is used to configure the MIDI port settings.
\screenshot{midi_ports_menu.png}\\
\begin{tabular}{|l|l|}
\hline
\rowcolor[HTML]{C0C0C0} 
Entry                                  & Function                                                                       \\ \hline
TURBO 1: {[} 1x, 2x, 4x, 8x {]}        & Port 1's Turbo MIDI speed                                                             \\ \hline
TURBO 2: {[} 1x, 2x, 4x, 8x {]}        & Port 2's Turbo MIDI speed                                                             \\ \hline
TURBO USB: {[} 1x, 2x, 4x, 8x {]}      & USB Port's Turbo MIDI speed                                                           \\ \hline
Driver 2: {[} Gener, Elekt {]}         &  Port 2's MIDI driver, either Generic or Elektron                                   \\ \hline
CTRL Port: {[} 2, USB, 2 + USB) {]}    & Which MIDI port provides control input (Note + CC)\\& for Sequencer + Chromatic pages.  \\ \hline                                    
\end{tabular}
\\\\
\textit{If using the Machinedrum MK1, set Turbo Speed no greater than 4x.}\\\\
\textit{{If you are intending to pair your Megacommand with a supported Elektron device}
(MNM, A4) set \textbf{Driver 2}  to Elektron. For all other MIDI devices (including unsupported Elektron machines), set \textbf{Driver 2} to General MIDI (default setting).}\\\\
\textit{If you are intending to use the paired Elektron device's MIDI thru to control further external equipment, you must set Turbo 2 speed to 1x.}

\newpage

\section{Sync:}
The Sync Config menu is used to configure MIDI clock + transport receive and destination settings.
\screenshot{midi_sync_menu.png}\\
\begin{tabular}{|l|l|}
\hline
\rowcolor[HTML]{C0C0C0} 
Entry                                  & Function                                                                       \\ \hline
CLOCK RECV: {[}1, 2, USB{]}                & Receive MIDI Clock from port.                                                  \\ \hline
TRANS RECV: {[}1, 2, USB{]}                & Receive MIDI Transport from port.                                               \\ \hline
CLOCK SEND: {[}OFF, 2, 2 + USB{]}          & Forward MIDI Clock to selected port(s)                                            \\ \hline
TRANS SEND: {[}OFF, 2, 2 + USB{]}          & Forward MIDI Transport to selected port(s)                                        \\ \hline
\end{tabular}
\\\\
\textit{Note: MCL does not generate its own MIDI Clock, instead it relies on a clock signal from either Port 1, Port 2 or USB.}
\\
\textit{The MD's internal clock settings are controlled by MCL and will be automatically configured based on the options selected above.}

\section{Routing:}
The MIDI Routing configuration menu can be used to set MIDI traffic forwarding between ports.
\screenshot{midi_route_menu.png}\\
\begin{tabular}{|l|l|}
\hline
\rowcolor[HTML]{C0C0C0} 
Entry                                  & Function                                                                       \\ \hline
MIDI1 FWD: {[}OFF, 2, USB, 2 + USB{]}                & Forward non-realtime MIDI data from Port 1 input \\ & to the selected MIDI output port(s).                                                \\ \hline
MIDI2 FWD: {[}OFF, 1, USB, 1 + USB{]}                & Forward non-realtime MIDI data from Port 2 input \\ & to the selected MIDI output port(s).                                               \\ \hline
USB FWD: {[}OFF, 1, 2, 1 + 2{]}          & Forward non-realtime MIDI data from USB-MIDI \\ & input to the selected MIDI output port(s).                                            \\ \hline
CC LOOP: {[}OFF, 2->2{]}          & Loopback MIDI CC messages on same port.                                      \\ \hline
\end{tabular}


\newpage
\section{Program:}
The Program menu is used to enable the sending and receiving of MIDI Program Change messages.
\screenshot{midi_prog_menu.png}\\
\begin{tabular}{|l|l|}
\hline
\rowcolor[HTML]{C0C0C0} 
Entry                                  & Function                                                                       \\ \hline
PROG Mode: {[}BASIC, ADV{]}                & Basic or Advanced. 
\\ \hline
PRG IN: {[}--, 1 .. 16, OMNI{]}            & Program change receive channel.                                            \\ \hline
PRG OUT: {[}--, 1 .. 16{]}          & Program change transmit channel.                                          \\ \hline
\end{tabular}
\begin{itemize}
    \item{\textbf{PROG Mode:} MIDI Program Change IN has two modes of operation, BASIC or ADV (Advanced).
\begin{itemize}
    \item{ \textbf{BASIC:} Program Change Receive will "Group Load" an entire row according to Load Page mode and Group Select settings. }
    \item{  \textbf{ADV:} Program change receive will set the row for slots to be loaded from.\\\\ MIDI notes from C3 upwards on Port 2 can be used to select slots to be loaded.\\\\  In this way, you can load any slot (or multiple slots) using a combination of Program Change + MIDI note. \\\\
 If Program Change is not received before MIDI note on/off, MCL will load from the current selected GUI row.\\\\ 
 MCL cannot load instantaneously, there is a minimum 1 bar delay.}
\end{itemize}}
\item{\textbf{PRG In:}
Set the receive channel for Program Change and Note messages on Port 2.\\\\
When set to -- (default) Program Change In is disabled.}
\item{\textbf{PRG Out:}
Set the send channel for Program Change messages on Port 2, which are sent when loading slots or rows.\\\\
When set to -- (default) Program Change Out is is disabled.}
\end{itemize}

\newpage

\section{MD MIDI:}
The MD MIDI configuration menu is used to configure Machinedrum driver's MIDI settings.
\screenshot{midi_md_menu.png}\\
\begin{tabular}{|l|l|}
\hline
\rowcolor[HTML]{C0C0C0} 
Entry                                  & Function                                                                       \\ \hline
CHRO CHAN: {[}--, 1..16{]}        & MIDI Channel for MD Chromatic mode. \\ \hline
POLY CHAN: {[}--, 1..16{]}         & MIDI Channel for MD Polyphonic mode. \\ \hline
TRIG CHAN: {[}--, 1..16{]}        & MIDI Channel for MD Trig mode. \\ \hline
\end{tabular}

\begin{itemize}
\item{\textbf{CHRO CHAN:} The CHRO CHAN setting is used to control which input channel the MD should receive note data from when in Chromatic mode. When set to INT (default) the MD will be controlled by the MD's \textbf{[Trig]} keys in chromatic mode. \\\\\textit{For this setting to work correctly, CTRL PORT in PORT CONFIG must also be set to the desired port.}}

\item{\textbf{POLY CHAN:} Dedicated MIDI Channel for playing designated Polyphonic tracks chromatically.}
\item{\textbf{TRIG CHAN:} Dedicated MIDI channel for triggering MD tracks via Note On message. Useful for triggering sounds using a MIDI drum pad. Track Mapping starts from note C2. Note Velocity is mapped to volume.}
\end{itemize}

\section{General MIDI:}
The General MIDI configuration menu is used to configure the Generic MIDI driver's settings.
\screenshot{general_md_menu.png}\\
\begin{tabular}{|l|l|}
\hline
\rowcolor[HTML]{C0C0C0} 
Entry                                  & Function                                                                       \\ \hline
MUTE CC: {[} 0..127,--{]}        & CC parameter for sending/receving Ext MIDI track\\& mute state on Mixer Page. \\ 
\hline
\end{tabular}