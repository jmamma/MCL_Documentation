\chapter{Configuration Menu}
The Configuration Menu is used for project management and to change a variety of software and hardware settings.

\screenshot{config_menu_1.png}

\textit{The Configuration menu can be opened via the GridPage by pressing the \textbf{<Save> and <Load>} buttons simultaneously. Alternatively, you can use the MD's \textbf{[Bank Group] and [Global]} buttons.}\\\\
To enter sub-menus press the \textbf{<YES>} button. Press the \textbf{<NO>} button to go back one level or exit the menu.
\section{Load Project:}
The Load Project sub-menu will display a list of MCL Projects that are stored on the root folder of the Micro SD card.\\\\
The current project is always selected first and is indicated by an '>' character next to its name.
\screenshot{project_menu.png}
%\fbox{\includegraphics[scale=.40]{project_menu.png}}\\

\subsection{Delete or Rename Project:}
From the file options menu, you can delete or rename projects.\\\\
\textit{From within the Load Project page, press and hold \textbf{<Shift>/[Global]} to access the file options menu.\\
Use the encoder to make your selection, release \textbf{<Shift>/[Global]} to activate your choice.}

\newpage
\section{New Project:}
The New Project options loads the New Project Page. This page allows you to specify a name for a project which will then be created.\\
%\fbox{\includegraphics[scale=.40]{new_project.png}}\\
\screenshot{new_project.png}

\screenshot{charpane.png}
\textit{All text editing pages in MCL allow access to char pane. Hold \textbf{<Page>/[Function].}}

\section{Project Files:}
Projects are stored on the SD Card in the \textbackslash{}Projects\textbackslash{} directory.
Each project has its own folder, inside the project folder there are 3 files:
\begin{verbatim}
\Projects\new_project_000\
                          |- new_project_000.mcl   <- Project master file
                          |- new_project_000.0     <- Grid X data
                          |- new_project_000.1     <- Grid Y data
\end{verbatim}

\section{Saving Projects:}
While loading projects recalls all saved track information, saving projects is not necessary. The project file is updated whenever slots or tracks are saved in the grid.