\chapter{Machinedrum Settings Menu:}
The Machinedrum settings are used to control how the MCL firmware interacts with the MD.
\screenshot{machinedrum_menu.png}
\section{Normalize:}

When the NORMALIZE option is set to AUTO (default), all saved MD tracks have their LEV boosted to 127, and parameters controlling VOL (including parameter locks) are lowered
to compensate. LFOs with destination VOL are 
adjusted too.\\
\\
The resulting track loudness remains the same, but the Track LEV parameter is no longer set arbitrarily. The maximum value of LEV (127), will always be the loudest volume for each track.\\
\\
When the sequencer is running, the LEV paramater is never transmitted to the MD. This allows the performer to fade tracks in and out of a mix using the LEV parameter.\\
\\
The performer can then confidently raise the LEV of the track to 127, knowing this is the is the intended maximum loudness for the track.
\section{CTRL Chan:}
The CTRL CHAN setting is used to control which input  source the MD should use for note data when in Chromatic mode.

When set to INT (default) the MD will be controlled by the MD's \textbf{[Trig]} keys in chromatic mode.

If set to a specific MIDI channel number, the MD will be controlled by node data sent that channel via port 2. If set to OMNI the MD will be controlled by any channel on port 2.
\section{Poly Config}
The POLY CONFIG option opens the Voice Select page and allows 2 or more voices of the MD to act as voices of a polyphonic synthesizer.