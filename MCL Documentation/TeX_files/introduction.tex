\chapter{Introduction}

\section{Preface}

\begin{small}
\textbf{\textit{The following document is intended as an operating manual for the MegaCommand Live firmware. For instructions on how to upload the firmware, or to learn how to build a MegaCommand MIDI controller please see the links below \footnote{MegaCommand documentation: \url{https://github.com/jmamma}.}.}}
\end{small}

\section{Hello}
Welcome to MegaCommand Live, 
\\
\\
MegaCommand Live (MCL) is a firmware designed for the MegaCommand MIDI controller that enhances the Elektron Machinedrum's (MD) sequencing, sound design and live performance capabilities.
\\
\\
At the heart of MCL lies the ability to retrieve individual tracks from the MD and store them on the MegaCommand (MC) for later playback. Chain Mode, inspired by classic music trackers, allows multiple tracks to be linked together to form complex musical phrases. During a performance, tracks from different patterns can be chained and mixed together.
\\
\\
MCL features a modern sequencer replacement for the Elektron Machinedrum. It consists of 22 local sequencer tracks with individual track lengths, parameter locks, micro-timing and conditional trigs. Sixteen of these tracks can be used to sequence the MD, the remaining six tracks are polyphonic and can be used to sequence an external MIDI device.
\\
\\
Other features of the MCL firmware include: MD Interface Integration, Chromatic + Polyphonic Mode,  Level Mixer, Audio Mute + Routing System, Sound Browser, Single Cycle Waveform Designer, Global LFO, FX Pages, Automated RAM recording, Turbo 8x MIDI and much more.
\\
\\
A great deal of care has gone in to designing the Graphical User Interface, the inter connectivity with the MD's hardware and operating system. The firmware has been highly optimised to facilitate outstanding MIDI performance and stability at TURBO speeds.
\\
\\
We have intended to make the firmware as intuitive as possible, particularly for users already familiar with the MD. Please take your time to read each section carefully. 
\\
\\
Let's get started.