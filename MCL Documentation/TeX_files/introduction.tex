\chapter{Introduction}

\section{Preface}

\begin{small}
\textbf{\textit{The following document is intended as an operating manual for the MegaCommand Live firmware. For instructions on how to upload the firmware, or to learn how to build a MegaCommand MIDI controller please see the links below \footnote{MegaCommand documentation:\url{https://github.com/jmamma}.}.}}
\end{small}

\section{Hello}
Welcome to MegaCommand Live, 
\\
\\
MegaCommand Live (MCL) is a firmware designed for the MegaCommand MIDI controller that enhances the Elektron Machinedrum's (MD) sequencing, sound design and live performance capabilities.
\\
\\
At the heart of MCL lies the ability to retrieve individual Tracks from the MD and store them on the MegaCommand (MC) for later recall. Chain Mode, inspired by classic music trackers, allows multiple tracks to be linked together to form complex musical phrases. During a performance, tracks from different patterns can be loaded and mixed together.
\\
\\
MCL features a modern sequencer replacement for the Elektron Machinedrum. It consists of 20 local sequencer tracks with individual track lengths, parameter locks, micro-timing and conditional trigs. Sixteen of these tracks can be used to sequence the MD, the remaining four tracks are polyphonic and can be used to sequence external instruments such as the Elektron Analog 4. 
\\
\\
Other features of the MCL firmware include: MD Trigger Interface, Chromatic + Polyphonic Mode,  Level Mixer, Audio Mute + Routing System, Sound Browser, Single Cycle Waveform Designer, Global LFO, FX Pages, Automated RAM recording, Turbo 8x MIDI and much more.
\\
\\
A great deal of care has gone in to designing the Graphical User Interface, the inter connectivity with the MD's hardware and operating system. The firmware has been highly optimised to facilitate outstanding MIDI performance and stability.
\\
\\
There is a great deal of depth to MCL. Please take your time to read each section carefully.