\chapter{Step Editor Page}
The Step Editor (StepEdit) page is used to program MCL's internal sequencer for a selected MD track and is fully integrated with the MD's pattern editing user interface.
\screenshot{step.png}\\
\textit{Press \textbf{[Bank Group] + [Trig 5]} key to open the StepEdit page.}\\\\
\textit{Pressing the MD's \textbf{[Rec]} button will allow you to toggle in and out of the StepEdit page from anywhere within MCL.}
%\fbox{\includegraphics[scale=.40]{seq_step_page.png}}
\screenshot{step_action.png}

\encodersbuttons{Trig Condition}{Micro-Timing}{Track Length}{Note}{Toggles REC}{}{Rotate Sequencer Page | Apply All}{Track Menu}
\newpage
\section{GUI}
\begin{itemize}
\item The StepEdit page leverages the MD's GUI for editing.
\item Trig editing mode can be switched between Lock, Mute and Slide by pressing \textbf{[Function] + [Bank B/C/D] }respectively .
\item On MCL the 16 steps of the current page for the current track are displayed on the bottom row.
\item The \textbf{[Trig]} buttons on the MD correspond to the visible steps. Pressing a \textbf{[Trig]} will add the step. A press followed by quick release will remove the step. 
\item Trig Conditions and Micro-Timing settings are per step and are set by holding \textbf{[Trig]} and pressing the MD's \textbf{[Up/Down]} keys to change Trig Conditions and \textbf{[Left/Right]} for Micro-Timing.
\item Parameter locks can be added to a step by selecting a \textbf{[Trig]} and then rotating the corresponding MD \textbf{[Encoder]}.
\item When a parameter lock is set, the step LED will blink on the MD, and the MCL will display a filled rectangle above step trig on screen.
\item Trigless locks are possible by pressing a \textbf{[Trig]} with parameter lock data, the step LED will become unlit and blink, signifying only lock data will be sent.
\item Parameter locks can be removed individually by selecting a \textbf{[Trig]} and tapping the MD's matching \textbf{[Encoder]}. 
\item Hold the \textbf{[Trig]} and press \textbf{[Clear]} to clear a step.
\item Parameter lock transmission can be disabled for any step by removing the lock step in LOCK editor mode.
\item Trigs and parameter lock transmission can be temporarily disabled by adding a step in MUTE editor mode.
\item A step can be muted/unmuted by holding down the matching \textbf{[Trig]} key and pressing \textbf{[Exit/No]}.
\item A step can be previewed by holding down the matching \textbf{[Trig]} key and pressing \textbf{[Yes]}.
\item The visible sequencer page can be toggled by pressing or the MD's \textbf{[Scale]} key.
\item The steps of the sequence can be shifted left or right by holding the MD's \textbf{[Function]} key and pressing \textbf{[Left]} or \textbf{[Right]}
\item The MD's \textbf{[Clear/Copy/Paste]} functions work across Track, Sequencer Page and Step.
\item The Length and Speed of the track can be set via the MD's Scale menu by pressing \textbf{[Function]} followed by \textbf{[Scale]}.

\end{itemize}
\newpage
\section{Trig Conditions}
\begin{itemize}
\item L1,L2,L3,L4,L5,L6,L7,L8 (For Ln, step is only triggered after every n iterations of track)
\item P10, P25, P50, P75, P90 (For Pxx, step has a xx percent chance of being triggered)
\item 1S. (One Shot trig, step is only triggered once after load)
\item Each trig condition above has a twin denoted by the \^{} character e.g L1\^{}, L2\^{}, P1\^{}. When these condition modes are selected, parameter locks and slides must also obey the trig condition.
\end{itemize}
\section{Micro-Timing}
When editing Micro-timing, the uTIMING window will be visible on both the MC and MD. The number of divisions between notes is dependent on the track's Speed setting.\\\\\textit{The MC's sequencer has a resolution of 1/192th per quarter notes}\\
\screenshot{utiming1.png}
\section{Track Speed \& Length}
All sequencer tracks can be played at an independent speed and length.\\\\
The chosen speed can be one of either: 1x, 2x, 3/2x, 3/4x, 1/2x, 1/4x, 1/8x.\\Triplets can be achieved using either 3/2x, 3/4x.
\begin{itemize}
\item The MD's "Scale Setup" Menu accessible via \textbf{[Func] + [Scale]} can be used to quickly change the track speed \& length. If the "Scale Setup" menu is opened outside of Step Edit mode, the speed \& length changes will apply to all 16 tracks.
\end{itemize}
\newpage
\section{Change Edit Mode}
By default, the Step Edit page opens in Trig Edit mode. It is possible to change this edit mode to program the track's sequencer data for either Trigs, Locks, Slides or Mutes.
\begin{itemize}
\item Edit mode can be toggled between Trig, Lock, Mute and Slide by pressing the MD's \textbf{[Func] + [Bank B/C/D]} keys.
\end{itemize}
\section{Clearing a Sequence}
\begin{itemize}
\item Enter [\textbf{Func]} + \textbf{[Clear]} to clear the current track when [\textbf{Rec}] LED lit, or all tracks when [\textbf{Rec]} LED unlit. (repeat the input to UNDO)
\end{itemize}
\section{Rotate Track Page}
Each track consists of 4 pages of 16 steps, for a total of 64 steps per track.
\begin{itemize}
\item The MD's \textbf{[Scale]} button can be used to rotate the visible track page.
\end{itemize}
\section{Chromatic Step Edit}
The Step Edit allows you to adjust a step's pitch by setting a note value. 
\begin{itemize}
\item Press and hold trigger button(s) on the MD. Adjusting \textbf{<Encoder 4 >} will allow you change the note value of the selected steps.
\item An external MIDI keyboard connected on Port 2, can be used to select the note.
\item A keyboard will be drawn on the display, showing the current note.
\end{itemize}
\screenshot{step_keyboard.png}
\section{Slides}
Each parameter lock can be made to slide up or down to the nearest step containing a parameter lock of the same type.
\\\\
To add slides to your sequence change the Edit Mode to "SLIDE" (see above).
\section{Mutes}
Individual steps of your sequence can be muted, by placing a mute trig at the corresponding position. Mutes are reset on sequencer restart.\\\\ 
To add mutes to your sequence change the Edit Mode to "MUTE" (see above).\\\\
Alternatively, from Pattern Edit, a step can be muted/unmuted by holding down the matching \textbf{[Trig]} key and pressing \textbf{[Exit/No]}.

\section{Shift Sequence}
The sequence of an individual track can be shifted left or right. Press \textbf{[Function]} + \textbf{[Left/Right]} on the MD. \\\\You can also shift the entire pattern left or right from the SHIFT menu via [\textbf{Global}].\\\\
Similarly you can reverse a track's sequence by pressing [\textbf{Function}] + [\textbf{Up}]
\section{Live Record}
Live Record mode can be activated using the MD's \textbf{[Rec]} + [\textbf{Play}] function.  The [\textbf{Rec}] LED will then blink to signify live recording is active. Both trig and parameter locks can be recorded simultaneously.\\
To record from an external DrumPad or Keyboard:
\begin{itemize}
    \item Ensure that the \textbf{Config-->MIDI-->CTRL PORT} is set to the MIDI port your external DrumPad/keyboard is connected.
    \item Set \textbf{Config-->MIDI-->MD MIDI-->TRIG CHAN} from INT (internal) to a desired MIDI channel or OMNI (all channels).
\end{itemize}
Press [\textbf{Rec}] to end live recording and return to step edit mode.

\newpage