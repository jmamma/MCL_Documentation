\chapter{Key Concepts}

\begin{itemize}
\item Page:
\\
The MCL firmware consists of pages accessible through the MC's function and encoder buttons. Each page contains unique behaviour and is described in this manual.
\item Project:
\\
A project stored on the Micro SD-Card. Each project contains one Grid. 
The maximum number of projects is only limited by the SD Card capacity.

\item Grid:
\\
The MCL Firmware uses a Grid/Slot system to store Tracks. 
The grid dimensions are 128 Rows x 20 Slots. 

\item Row/Pattern:
\\
A row of the Grid. Each row consists of 20 slots and can store 16 MD tracks, a complete MD pattern + 4 external sequencer or A4 tracks.

\item Slot:
\\
A position in the Grid where a Track can be stored. (Either occupied or unoccupied).

\item Tracks
\\
A track copied from the Elektron MD or AnalogFour, or an external MIDI track.

There are 3 types of tracks.
\begin{itemize}

\item MachineDrum Track (Slots 0-15):

A track merged from the MD containing: Machine Settings, Internal Sequencer Data, Master Effects Settings.
\item AnalogFour Track (Slots 16-19):

A track merged from the A4 containing: Sound Settings, Internal Sequencer Data
\item Ext Track (Slots 16-19):

A General MIDI device track only containing Local Sequencer Data.
\end{itemize}

\item Internal Sequencer:\\
MCL's 20 track internal sequencer.

\item Basic Workflow:\\
The Grid is similar to the banks of the MD, but different. An active pattern is first chain-written from the Grid into the Internal Sequencer and then ready for playback. 

Different tracks from different rows in the Grid can be mix-and-matched into the Internal Sequencer too.

Patterns created in the Sequencer Pages are kept in the Internal Sequencer, and thus not immediately saved to the Grid, until the MCL firmware is told so in the Save Page. 

A common gotcha is to create a pattern in the Sequencer Pages, use the Chain Write page to load data from the Grid, only to find the just-created pattern gone. To avoid this, make sure to \textbf{save modified internal sequence before chain write}.

\end{itemize}


