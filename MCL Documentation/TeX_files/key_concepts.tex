\chapter{MegaCommand Hardware}
The MegaCommand is an Open Source MIDI controller developed to run the MCL OS. It is currently available in two form factors, the MegaCommand DIY (2017) and the MegaCMD (2021). Both were designed by Justin Mammarella.\\\\
The MegaCommand DIY is built upon the Arduino Mega 2560 development board; requiring skills in both soldering and self assembly. The MegaCMD is a pre-built version based on a new SMD design, requiring no user assembly.\\\\
The MegaCMD benefits from some additional circuitry that allows USB disk access to the MicroSD card for file transfer between a host computer.\\\\
The MegaCommand DIY can be powered via a standard DC power jack whilst the MegaCMD only accepts power via USB.
\\\\
The MiniCommand (2010) is an older controller developed by Ruin \& Wesen that paved the development of the MegaCommand, it is not compatible with the current version of MCL.
\chapter{Terminology and Conventions}
   \begin{itemize}
      \item MCL - MegaCommand Live OS.
      \item MD - Machinedrum
      \item MDX - Machinedrum X OS.
      \item MC - MegaCommand or MegaCMD MIDI controller
      \item \textbf{<Button> } Enclosed arrows will reference the use of a MegaCommand function button or encoders to perform an action.
      \item \textbf{[Key] } Enclosed square brackets will reference the use of a Machinedrum key to perform an action. 
   \end{itemize}


\newpage
\chapter{Key Concepts}

\begin{itemize}
\item \textbf{Page}
\\
The MCL firmware consists of pages accessible through either the MC's or MD's GUI. Each page contains distinct functionality and is described in this manual.
\item \textbf{Project}
\\
A project stored on the Micro SD-Card.
The maximum number of projects is only limited by the SD Card capacity.

\item \textbf{Grid}
\\
The MCL Firmware uses a Grid \& Slot system to store Tracks.\\
Each project contains two grids, X and Y. Grid dimensions are 16 Slots x 128 Rows.\\
\\
Grid X is used to store 16 MD tracks.\\Grid Y is used to store 6 External MIDI tracks + 5 AUX tracks.
\item \textbf{Bank:}\\
The rows of the Grid X are divided in to groups of 16, forming 8 banks A,B,C,D,E,F,G,H.
\item \textbf{Row/Pattern}
\\
A row of the Grid.

\item \textbf{Slot}
\\
A position in the Grid where a Track can be stored. (Either occupied or unoccupied).
\item \textbf{Track}
\\
An MCL sequencer track that may contain both sound and MIDI sequencer data.

There are 3 types of tracks.
\begin{itemize}

\item \textbf{MachineDrum Track} (Grid X: Slots 1-16):
A sequencer track for the Elektron MD. Each MD Track contains the Machine's Sound Settings and Sequencer Data.

\item \textbf{External MIDI Track} (Grid Y: Slots 1-6):
A polyphonic sequencer track used to control a sound module connected via MIDI. Each External MIDI track contains Sequencer Data, and for supported Elektron devices sound data is retained. 

\item \textbf{AUX Track} (Grid Y: Slots: 12-16)\\
An auxiliary track. Used to store/recall one of: Performance Controllers + Mute Sets, MCL's LFO settings, audio Routing,  the Machinedrum's master FX settings and Tempo settings.
\end{itemize}

\item \textbf{Device}\\
An attached MIDI device on MIDI port 1 or 2.
\item \textbf{Group}
\\
Slots are categorised into groups based on the device they belong to, or a shared characteristic.
\end{itemize}